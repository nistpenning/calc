\documentclass[11pt, oneside,reqno]{amsart}   	% use "amsart" instead of "article" for AMSLaTeX format
\usepackage{geometry}                		% See geometry.pdf to learn the layout options. There are lots.
\geometry{letterpaper}                   		% ... or a4paper or a5paper or ... 
%\geometry{landscape}                		% Activate for for rotated page geometry
%\usepackage[parfill]{parskip}    		% Activate to begin paragraphs with an empty line rather than an indent
\usepackage{graphicx}				% Use pdf, png, jpg, or eps§ with pdflatex; use eps in DVI mode
								% TeX will automatically convert eps --> pdf in pdflatex		
											
\setcounter{secnumdepth}{-2}
\usepackage{amssymb}
\usepackage{amsmath}
\usepackage{braket}
%\usepackage{mathtools}

\newcommand{\gausslaw}{ \oint \vec{E} \cdotp d\vec{A} = \frac{Q_{encl}}{\epsilon_0}}
\newcommand{\eqn}[1]{\begin{equation}#1 \end{equation}}
\renewcommand{\vec}[1]{\textbf{#1}}
\renewcommand{\bold}[1]{\textbf{#1}}
\newcommand{\tperp}{$T_{\perp}$}

\title{ Scattering Rate, Energy Balance, and Torque \\ June 9$^{th}$ 2015}
\author{Steven Torrisi }

\begin{document}
\maketitle
\emph{Described below is a derivation of expressions used in the Wayne Itano code for predicting the steady-state equilibrium temperature, net torque, and total scattering events of a doppler cooling laser incident on an ion plasma in a penning trap.}
%
%\emph{Note: We will indicate whenever constant coefficients unimportant to the integration are dropped or other substitutions are made.} \\
%
%\textbf{Functions:}
%
%Intensity: $ I(y)= I_0 e^{-(y-d)^2 /\omega_y^2} $ 
%
%Density: $ \rho (x,y)= \Sigma_0 \sqrt{1- \frac{x^2+y^2}{R^2}}$ , if  $ x^2+y^y \leq R^2;   0$  otherwise.
%
%Scattering Energy: $ E(v_x, y) = \hbar k (v_x - \omega_r y) + 2R $
%
%Scattering Rate: $ \gamma_L= \frac{I \sigma_0}{\hbar \omega_L} \frac{ \gamma_0^2 /4}{\gamma_0^2 /4 + \Delta^2}$
%
%\textbf{Useful Constants}
%
%$vk = \frac{\gamma_0}{2 k}$ 
%
%To be completed later...

\section{Total Scattering Rate}

In order to calculate the total scattering rate, we begin with the expression for the scattering rate from Itano 1988:
\eqn{\gamma_L = \frac{I \sigma_0}{\hbar \omega_L} \frac{\gamma_0^2/4}{(\gamma/2)^2 + \Delta^2}}

Where I is the laser intensity function, $\sigma_0$ is the cross-section/probability of interaction, $\omega_L$ is the laser frequency, $\gamma_0$ is the transition line width, and $\Delta$ is the laser detuning.

This scattering rate thus depends on two \emph{functions}; intensity $I$, and $\gamma$ modified for saturation effects.

The intensity is the beam profile, which we write here as 
\eqn{ I= I_0 \exp[ -2(y-d)^2/\omega_y^2]}
assuming a Gaussian beam profile with offset d and beam width $\omega_y$. If we are dealing with a 2-dimensional ion crystal, we neglect the z-dependence of the laser beam intensity.

The saturation-adjusted $\gamma$ is written more simply in the $\gamma^2$ form as  \eqn{ \gamma^2 = \gamma_0^2 (1+2S)}
where S is a saturation parameter defined as
\eqn{S= \frac{S_0}{I_0} I = \frac{I_0 \sigma_0}{\hbar \omega_0 \gamma_0}  \exp[ -2(y-d)^2/\omega_y^2]}
noting \eqn{S_0 =  \frac{I_0 \sigma_0}{\hbar \omega_0 \gamma_0} }
After some simplification we can write the scattering rate $\gamma_L$ as

\eqn{ \frac{\gamma_0 S_0 \exp [ -2 (y-d)^2/\omega_y^2] }{\left[ 1+2S_0 \exp [ -2 (y-d)^2/\omega_y^2] +\left( \frac{2}{\gamma_0} \right)^2 \Delta^2 \right]}}

Now, we can weight this total scattering rate $\gamma_L$ over the entire cloud by multiplying it by density,
 \eqn{ \text{ If $x^2+y^2 \leq R_p^2$, then } n(x,y) = \Sigma_0 \sqrt{1-\frac{x^2+y^2}{R_p^2}} }
\eqn{ \text{else, } n(x,y)= 0}

Defining the density function in this piecewise way is suggested for numerical studies, as it permits rectangular integration bounds ($[-R_p,R_p]$ for and $y$ as opposed to integrating y from $\pm \sqrt{R_p^2-x^2}$, for example).

We additionally need to integrate over $v_x$ to account for Doppler shift effects in the laser detuning (a crucial physical effect in the math).

The total scatter rate can be written then as
\eqn{ \int_{-R}^{R} dx \int_{-R}^R  dy \int_{-\infty}^{\infty} dv_x \gamma_L n(x,y) P(v_x| y, u,\omega)}
where $P(v_x| y, u,\omega)$ represents a Maxwell-Boltzmann velocity distribution for the x component of the velocity, with mean parameter $u$ and dependence on $\omega y$ to account for weighting over the spinning cloud.  Recalling that  \eqn{\Delta= \omega_L - \omega_0 - \frac{R}{\hbar} - kv_x}
continue on:
\eqn{ \gamma_0 S_0 \Sigma_0
 \int_{-R}^{R} dx \int_{-R}^R  dy \int_{-\infty}^{\infty} dv_x 
 \frac{
 \exp [ -2 (y-d)^2/\omega_y^2] \sqrt{1-\frac{x^2+y^2}{R_p^2}} P(v_x| y, u,\omega)}
 { \left[ 1+2S_0 \exp [ -2 (y-d)^2/\omega_y^2] +\left( \frac{2}{\gamma_0} \right)^2 \Delta^2 \right]}
 }
or expanding $P(v_x)$ and $\Delta$,
\eqn{ \gamma_0 S_0 \Sigma_0
 \int_{-R}^{R} dx \int_{-R}^R  dy \int_{-\infty}^{\infty} dv_x 
 \frac{
 \exp [ -2 (y-d)^2/\omega_y^2] \sqrt{1-\frac{x^2+y^2}{R_p^2}} \exp [- (v_x-\omega y)^2 /u^2]
 }
 { \left[ 1+2S_0 \exp [ -2 (y-d)^2/\omega_y^2] +\left( \frac{2}{\gamma_0} \right)^2  (\omega_L - \omega_0 - \frac{R}{\hbar} - kv_x)^2 \right]\sqrt{\pi} u}
 }
 
 Make the substitution $v'= v_x-\omega_r y$ and redefine $\omega_l-\omega_0$ as $\omega_0'$ to get

 \eqn{ \gamma_0 S_0 \Sigma_0
 \int_{-R}^{R} dx \int_{-R}^R  dy \int_{-\infty}^{\infty} dv' 
 \frac{
 \exp [ -2 (y-d)^2/\omega_y^2] \sqrt{1-\frac{x^2+y^2}{R_p^2}} \exp [-v'^2 /u^2]
 }
 { \left[ 1+2S_0 \exp [ -2 (y-d)^2/\omega_y^2] +\left( \frac{2}{\gamma_0} \right)^2  (\omega_0' - k(\omega_r y +v'))^2 \right]\sqrt{\pi} u}
 } 

and now if $v= \frac{v'}{u}, dv'=udv$ we will arrive at our final expression.\emph{ (Watch carefully; the u in the denominator is cancelled by $\frac{dv'}{u} = dv$)}

 \eqn{ \gamma_0 S_0 \Sigma_0
 \int_{-R}^{R} dx \int_{-R}^R  dy \int_{-\infty}^{\infty} dv 
 \frac{
 \exp [ -2 (y-d)^2/\omega_y^2] \sqrt{1-\frac{x^2+y^2}{R_p^2}} \exp [-v^2]
 }
 { \left[ 1+2S_0 \exp [ -2 (y-d)^2/\omega_y^2] +\left( \frac{2}{\gamma_0} \right)^2  (\omega_0' - k(\omega_r y +v u))^2 \right]\sqrt{\pi}}
 } 
 
 Which when integrated over the cloud will yield the total scattering rate, accounting for saturation effects.

 \section{Torque}

 We need only modify the most recent equation to add in a factor for the torque imparted per scattering event.
 
 Because the momentum transfer per scattering event is given as $\hbar k$, where k is the wave number of the photon, we can multiply in the momentum incurred per event times the y coordinate (through $\tau = \vec{r} \times \vec{F}$ to get the following expression for total torque:
 
 \eqn{ \gamma_0 S_0 \Sigma_0
 \int_{-R}^{R} dx \int_{-R}^R  dy \int_{-\infty}^{\infty} dv 
 \frac{
 (\hbar k y)
 \exp [ -2 (y-d)^2/\omega_y^2] \sqrt{1-\frac{x^2+y^2}{R_p^2}} \exp [-v^2]
 }
 { \left[ 1+2S_0 \exp [ -2 (y-d)^2/\omega_y^2] +\left( \frac{2}{\gamma_0} \right)^2  (\omega_0' - k(\omega_r y +v u))^2 \right]\sqrt{\pi}}
 \label{torque}
 } 
 
 \section{Energy Balance}
 
 Finally, we may modify the total scattering rate integral so that it reflects the change in energy per second.
 
 Just as we introduced a factor into the integrand which represented the momentum imparted per scattering event for the torque, we will replace the $\hbar k y$ factor with a term representing the energy per scattering event.
 
 The functional purpose of this equation is to numerically solve for a root of $< \frac{dE}{dt}>$ so that we can find the steady-state temperature.
 
 Thus, we can throw away the constant coefficients found in equation \ref{torque} of $\gamma_0, S_0,$ and $ \Sigma_0$ since they will not affect a root. This makes it easier for a computer to compute the values of this integral.
 
 Replacing the momentum per scattering event $\hbar k y$ with the energy per scattering event, given by laser vector $\vec{k}$ and ion velocity $\vec{v}$ to be
 \eqn{ \hbar \vec{k} \cdot \vec{v} + 2 R} where R is a recoil energy term defined by  \eqn{R= \frac{ (\hbar k )^2}{2m}}.
 
  The laser is exclusively in the x direction, so we can rewrite this energy term as 
  \eqn{\hbar k v_x + 2R}
  
  and adjust for the doppler effect by replacing $v_x$ with
  \eqn{ \hbar k (v_x-\omega y)}
  
  Recall earlier we made a substitution which eliminated $v_x$ and replaced it with $v$; we will do the same here in a truncated fashion:
 \eqn{ v_x - \omega y \rightarrow v' \rightarrow vu }
 leaving us with 
 \eqn{ \hbar k uv + 2R}
 since we will be eventually solving for a zero, we may factor out  $\hbar k u $ from both terms and throw away the constant coefficients when factored into the integrand.
 
 We finally have \eqn{ v + \frac{2R}{\hbar k u} = v+ \frac{\hbar k}{m u}}
 
 Which we now substitute into the total scattering rate integral to get our final expression:

  \eqn{ 
 <\frac{dE}{dt}> = 
 \int_{-R}^{R} dx \int_{-R}^R  dy \int_{-\infty}^{\infty} dv 
 \frac{
 (v- \frac{\hbar k}{m u})
 \exp [ -2 (y-d)^2/\omega_y^2] \sqrt{1-\frac{x^2+y^2}{R_p^2}} \exp [-v^2]
 }
 { \left[ 1+2S_0 \exp [ -2 (y-d)^2/\omega_y^2] +\left( \frac{2}{\gamma_0} \right)^2  (\omega_0' - k(\omega_r y +v u))^2 \right]\sqrt{\pi}}
 \label{dedt}
  }
  which may be numerically solved for to find the steady-state temperature as a function of $u$ in the \textbf{Wayne/Itano code}.

\end{document}