\documentclass[12pt]{article}
\usepackage[left=2cm,top=2cm,bot=2cm,right=2cm,nohead,nofoot]{geometry}%bot=.2cm,

%\usepackage{amsmath}
%\providecommand{\e}[1]{\ensuremath{\times 10^{#1}}}
\begin{document}
\title{Penning Trap Lagrangian}
\author{Adam C. Keith}
\maketitle

Lagrangian in the lab frame:
\begin{eqnarray}
L &=& \sum^N_{i=1} \Bigg[\frac{1}{2} m_i \left(\dot{x}_i^2 + \dot{y}_i^2 + \dot{z}_i^2 \right) - \frac{1}{2} m_i \omega_z^2 \left(z_i^2 - \frac{x_i^2 + y_i^2}{2} \right) - \frac{1}{2}  m_i \omega_z^2  G V_w  \left[\left(x_i^2 - y_i^2 \right)\cos{2\omega t} - 2 x_i y_i sin{2\omega t}\right] - \nonumber \\
&&  \frac{eB}{2} \left(\dot{x}_i y_i - \dot{y}_i x_i \right) - \frac{ke^2}{2} \sum^N_{j \neq i}\frac{1}{r_{ij}} \Bigg]
\end{eqnarray}
with $\frac{\omega_z}{2\pi}$ the axial trapping frequency, $\frac{\omega_z}{2\pi}$ the crystal rotation frequency, $G$ a geometrical factor relating trap electrodes, and $V_w$ relating to the strength of the rotating wall potential. Moving to a frame rotating at $\omega$ we have the new Lagrangian:

\begin{eqnarray}
L &=& \sum^N_{i=1} \Bigg[\frac{1}{2} m_i \left(\dot{x}_i^2 + \dot{y}_i^2 + \dot{z}_i^2 \right) +  \frac{1}{2} m_i \left(2\omega - \frac{eB}{m_i}\right) \left(\dot{x}_i y_i - \dot{y}_i x_i \right) - \frac{1}{2}m_i\omega_z^2 z_i^2 +
 \nonumber \\
&&  \frac{1}{2} m_i \left(\omega^2 + \frac{\omega_z^2}{2} - \frac{eB\omega}{m_i} \right) \left(x_i^2+y_i^2\right) - \frac{1}{2}  m_i \omega_z^2  G V_w  \left(x_i^2 - y_i^2 \right) - \frac{ke^2}{2} \sum^N_{j \neq i}\frac{1}{r_{ij}} \Bigg]
\end{eqnarray}
with all coordinates in rotating frame. Let's write the characteristic length and time scales as $l_0^3 = \frac{ke^2}{\frac{1}{2}m_{Be} \omega_z^2}$ and $t_0 = \frac{1}{\omega_z}$ where $m_{Be}$ is the Beryllium ion mass. We also write the cyclotron frequency $\frac{eB}{m_{Be}} = \omega_c \omega_z$ and the rotation frequency as $\omega = \omega_r \omega_z$. With these definitions and substitutions we divide the Lagrangian by $\frac{ke^2}{l_0}$:

\begin{eqnarray}
L &=& \sum^N_{i=1} \Bigg[m_i \left(\dot{x}_i^2 + \dot{y}_i^2 + \dot{z}_i^2 \right) +  m_i \left(2\omega_r - \frac{\omega_c}{m_i} \right) \left(\dot{x}_i y_i - \dot{y}_i x_i \right) - m_i z_i^2 + m_i \left(\omega_r^2  + \frac{1}{2} - \frac{\omega_c \omega_r}{m_i} \right) \left(x_i^2+y_i^2\right) - \nonumber \\
&&  G V_w m_i \left(x_i^2 - y_i^2 \right) - \frac{1}{2} \sum^N_{j \neq i}\frac{1}{r_{ij}} \Bigg]
\end{eqnarray}
with all coordinates in dimensionless form and $m_i$ overloaded as fraction of Beryllium ion mass.

Now we expand the Lagrangian about the equilibrium positions in all variables. We assume at equilibrium all velocities are zero and force a planar structure so that all $z_i = 0$. This causes the expanded Lagrangian to separate in to a Lagrangian for the axial direction and one in the plane.
For the axial direction,
\begin{equation}
\frac{\partial L}{\partia z_{\alpha}} = -2 m_{\alpha}z_{\alpha}+  \sum^N_{j \neq \alpha}\frac{z_{\alpha}-z_j}{r^3_{\alpha j}}
\end{equation}

\begin{equation}
\frac{\partial L}{\partial \dot{z}_{\alpha}} = 2 m_{\alpha}\dot{z}_{\alpha}
\end{equation}
which both vanish at equilibrium as do the partials involving x and z or y and z (or $\dot{z}$). The only partials that don't vanish at equilibrium are

\begin{equation}
\frac{\partial L}{\partial \dot{z}_{\beta}} \frac{\partial L}{\partial \dot{z}_{\alpha}} = 2 m_{\alpha} \delta_{\alpha \beta}
\end{equation}

\begin{equation}
\frac{\partial L}{\partial z_{\beta}}\frac{\partial L}{\partial z_{\alpha}} = \delta_{\alpha \beta} \left[ -2 m_{\alpha}+\sum^N_{j \neq \alpha}\frac{r^2_{\alpha j} -3\left(z_{\alpha}-z_j \right)^2}{r^5_{\alpha j}}\right] - \left(1- \delta_{\alpha \beta} \right) \frac{r^2_{\alpha \beta} -3\left(z_{\alpha}-z_{\beta} \right)^2}{r^5_{\alpha \beta}}
\end{equation}
Thus, the expanded Lagrangian in the axial direction is

\begin{equation}
L = \frac{1}{2}\sum^N_{i=1} 2m_i \dot{z}_i^2 + \frac{1}{2}\sum^N_{i,j} z_i z_j \left( \delta_{i j} \left[ -2 m_{i}+\sum^N_{k \neq i}\frac{1}{\bar{r}^3_{i k}}\right]
-  \left(1- \delta_{i j} \right) \frac{1}{\bar{r}^3_{i j}} \right)
\end{equation}
where $z_i$ is overloaded to be the distance away from equilibrium position and $\bar{r}_{ji}$ is the equilibrium distance between ions.

% -----------------------------
We now Taylor expand the Lagrangian in $x_i$, $y_i$, $\dot{x}_i$, and $\dot{y}_i$. I'll write out all the derivatives explicitly:

\begin{equation}
\frac{\partial L}{\partial x_{\alpha}} = - m_{\alpha}\left[2\omega_r-\omega_c\right]\dot{y}_{\alpha} + 2m_{\alpha} \left(\omega^2_r +\frac{1}{2} - \frac{\omega_r \omega_c}{m_{\alpha}} - G V_w \right) x_{\alpha} +  \sum^N_{j \neq \alpha}\frac{x_{\alpha}-x_j}{r^3_{\alpha j}}
\end{equation}

\begin{equation}
\frac{\partial L}{\partial y_{\alpha}} = m_{\alpha}\left[2\omega_r-\omega_c\right]\dot{x}_{\alpha} + 2m_{\alpha} \left(\omega^2_r +\frac{1}{2} - \frac{\omega_r \omega_c}{m_{\alpha}} + G V_w \right) y_{\alpha} + \sum^N_{j \neq \alpha}\frac{y_{\alpha}-y_j}{r^3_{\alpha j}}
\end{equation}
The above derivatives vanish at equilibrium, by definition. The following two derivatives DO NOT vanish...
\begin{equation}
\frac{\partial L}{\partial \dot{x}_{\alpha}} = 2m_{\alpha}\dot{x}_{\alpha} + m_{\alpha}\left[2\omega_r - \omega_c\right]y_{\alpha}
\end{equation}

\begin{equation}
\frac{\partial L}{\partial \dot{y}_{\alpha}} = 2m_{\alpha}\dot{y}_{\alpha} - m_{\alpha}\left[2\omega_r - \omega_c\right]x_{\alpha}
\end{equation}


\begin{eqnarray}
\frac{\partial }{\partial x_{\beta}} \frac{\partial }{\partial x_{\alpha}} L &=& 
\delta_{\alpha \beta} \left[  2m_{\alpha} \left(\omega^2_r +\frac{1}{2} - \frac{\omega_r \omega_c}{m_{\alpha}} - G V_w \right) +\sum^N_{j \neq \alpha}\frac{r^2_{\alpha j} -3\left(x_{\alpha}-x_j \right)^2}{r^5_{\alpha j}} \right] + \nonumber \\
&& - \left(1- \delta_{\alpha \beta} \right) \frac{r^2_{\alpha \beta} -3\left(x_{\alpha}-x_{\beta} \right)^2}{r^5_{\alpha \beta}}
\end{eqnarray}

\begin{eqnarray}
\frac{\partial }{\partial y_{\beta}} \frac{\partial }{\partial y_{\alpha}} L &=& 
\delta_{\alpha \beta} \left[  2m_{\alpha} \left(\omega^2_r +\frac{1}{2} - \frac{\omega_r \omega_c}{m_{\alpha}} + G V_w \right) + \sum^N_{j \neq \alpha}\frac{r^2_{\alpha j} -3\left(y_{\alpha}-y_j \right)^2}{r^5_{\alpha j}} \right] + \nonumber \\
&& - \left(1- \delta_{\alpha \beta} \right) \frac{r^2_{\alpha \beta} -3\left(y_{\alpha}-y_{\beta} \right)^2}{r^5_{\alpha \beta}}
\end{eqnarray}

\begin{equation}
\frac{\partial }{\partial \dot{x}_{\beta}} \frac{\partial }{\partial \dot{x}_{\alpha}} L = 2m_{\alpha}\delta_{\alpha \beta} 
\end{equation}

\begin{equation}
\frac{\partial }{\partial \dot{y}_{\beta}} \frac{\partial }{\partial \dot{y}_{\alpha}} L = 2m_{\alpha}\delta_{\alpha \beta} 
\end{equation}

\begin{equation}
\frac{\partial }{\partial \dot{x}_{\beta}} \frac{\partial }{\partial x_{\alpha}} L = 0
\end{equation}

\begin{equation}
\frac{\partial }{\partial \dot{y}_{\beta}} \frac{\partial }{\partial y_{\alpha}} L = 0
\end{equation}

\begin{equation}
\frac{\partial }{\partial \dot{y}_{\beta}} \frac{\partial }{\partial x_{\alpha}} L = - m_{\alpha}\delta_{\alpha \beta} \left[2\omega_r - \omega_c \right]
\end{equation}

\begin{equation}
\frac{\partial }{\partial \dot{x}_{\beta}} \frac{\partial }{\partial y_{\alpha}} L =  m_{\alpha}\delta_{\alpha \beta} \left[2\omega_r - \omega_c \right]
\end{equation}

\begin{equation}
\frac{\partial }{\partial y_{\beta}} \frac{\partial }{\partial x_{\alpha}} L =
\delta_{\alpha \beta} \left[  -3\sum^N_{j \neq \alpha}\frac{\left(y_{\alpha}-y_j \right)\left(x_{\alpha}-x_j \right)}{r^5_{\alpha j}} \right] + \left(1- \delta_{\alpha \beta} \right) 3}\frac{\left(y_{\alpha}-y_{\beta} \right)\left(x_{\alpha}-x_{\beta} \right)}{r^5_{\alpha \beta}}
\end{equation}
Thus the expanded Lagrangian is:
\begin{equation}
L = \frac{1}{2}\sum^N_{i=1} 2m_i \left(\dot{x}_i^2 + \dot{y}_i^2 \right) +  m_i \left(2\omega_r - \omega_c \right) \left(\dot{x}_i(2 \bar{y}_i + y_i) - \dot{y}_i (2\bar{x}_i + x_i) \right) + \frac{1}{2}\mathbf{q}^T V \mathbf{q}
\end{equation}
%with Hamiltonian:
%\begin{equation}
%H = \frac{1}{2}\sum^N_{i=1} 2m_i \left(\dot{x}_i^2 + \dot{y}_i^2 \right) - \frac{1}{2}\mathbf{q}^T V \mathbf{q}
%\end{equation}

where $x_i$, etc is overloaded to be distance away from equilibrium position $\bar{x_i}$ and \textbf{q} is appropriate vector of positions to make sense with $V$ as matrix of derivatives (Hessian) (just a convenient way to write all the partials compactly). I do not quite understand the factor of two in front of $\bar{x}$ and $\bar{y}$ but doesn't seem to matter. Here are the planar equations of motion. 

\begin{equation}
\ddot{z}_i - \frac{1}{2} \sum^N_{j=1}  \frac{1}{m_i} z_j \left( \delta_{i j} \left[ -2 m_{i}+\sum^N_{k \neq i}\frac{1}{\bar{r}^3_{i k}}\right]
-  \left(1- \delta_{i j} \right) \frac{1}{\bar{r}^3_{i j}} \right) = 0
\end{equation}

\begin{equation}
\ddot{x}_i +\left(2\omega_r - \omega_c \right) \dot{y}_i - \frac{1}{2} \sum^N_{j=1} x_j \frac{V^{xx}_{ij}}{m_i} + x_j \frac{V^{xy}_{ij}}{m_i} = 0
\end{equation}

\begin{equation}
\ddot{y}_i -\left(2\omega_r - \omega_c \right) \dot{x}_i - \frac{1}{2} \sum^N_{j=1} y_j \frac{V^{yy}_{ij}}{m_i} + y_j \frac{V^{xy}_{ij}}{m_i} = 0
\end{equation}
$V_{ij}^{xy}$ is the second order partial with respect to $y_j$ then $x_i$. 

I will add to this how to get both planar and axial normal modes. Hopefully there were no typos or mistakes above!




\end{document}