\documentclass[12pt]{article}
\usepackage[left=2cm,top=2cm,bot=2cm,right=2cm,nohead,nofoot]{geometry}%bot=.2cm,

\usepackage{amsmath}
\providecommand{\e}[1]{\ensuremath{\times 10^{#1}}}
\begin{document}
\title{Comparing Trap Potentials}
\author{Adam C. Keith}
\maketitle

I write my trap potential like this:
\begin{equation}
\phi_i = V_0 \left( z_i^2 - \frac{x_i^2 + y_i^2}{2} \right) +  V_{wall}[(x_i^2 - y_i^2) cos2\theta - 2x y sin2\theta] 
\end{equation}
where I parametrize $V_{wall} = GV_wV_0$ where $G= 4.5 \e{-5}$ * (some units, probably V$^{-1}$) is some kind of geometric factor and $V_w$ is the actual 
potential applied the the rotating wall electrodes (the peak to peak potential, I think). I found an old conversation that John sent to me a long time ago
but I slightly changed the language:

"Following the notation used in the notes you sent me, one can write
\begin{equation}
qV_{o} = \frac{1}{2}m(w_z)^2 
\end{equation}
where $w_z=(2\pi)*(795 \text{kHz})$ and  following the notation in Hasegawa et al.,
\begin{equation}
qV_{wall} = [\frac{1}{2}m(w_z)^2]\delta
\end{equation}
where $\delta=(0.045)*(V_w/V_T)$.
In the past few months we have used two different rotating wall strengths corresponding to ($V_w/V_T$)= 0.035 (weak wall) and 0.105 (strong wall).
Multiplying these two ratios by the 0.045 geometric factor, the strength of the coefficient $qV_{wall}$ is 0.001575 (weak wall) and 0.004725 (strong wall) of the coefficient $qV_{o}$."


Dominic writes his potential like this:
\begin{equation}
\varphi(\mathbf{r}) = \frac{1}{2}k_z z^2 - 
\frac{1}{2} \left(k_x x_r^2 + k_y y_r^2\right)
\end{equation}
with
\begin{equation}
k_x=\left(\frac{1}{2}+\delta\right)k_z,\qquad 
k_y=\left(\frac{1}{2}-\delta\right)k_z\;,
\end{equation}
and
\begin{equation}
\left[
\begin{array}{c}
x_r\\
y_r
\end{array}\right] =
\left[
\begin{array}{cc}
\cos(\vartheta(t)) & -\sin(\vartheta(t))\\
\sin(\vartheta(t)) & \cos(\vartheta(t))
\end{array}\right]
\left[\begin{array}{c}
x\\
y
\end{array}\right]
\end{equation}
Lets expand this to compare to mine:
\begin{equation}
\varphi(\mathbf{r}) = \frac{1}{2}k_z z^2 - 
\frac{1}{2} \left[\frac{1}{2}k_z(x_r^2 + y_r^2) + \delta k_z (x_r^2 - y_r^2) \right]
\end{equation}

\begin{equation}
\varphi(\mathbf{r}) = \frac{1}{2}k_z z^2 - 
\frac{1}{2} \left[\frac{1}{2}k_z(x^2 + y^2) + \delta k_z\left(\left(\cos^2(\vartheta(t)) - \sin^2(\vartheta(t))\right) (x^2 - y^2) - 4xy\sin(\vartheta(t))\cos(\vartheta(t)) \right) \right]
\end{equation}

using trig identities 
\begin{equation}
\varphi(\mathbf{r}) = \frac{1}{2}k_z z^2 - 
\frac{1}{2} \left[\frac{1}{2}k_z(x^2 + y^2) + \delta k_z\left(\cos(2\vartheta(t))(x^2 - y^2) - 2xy\sin(2\vartheta(t)) \right) \right]
\end{equation}

Comparing, we have:
\begin{equation}
V_0 = \frac{1}{2}k_z
\end{equation}
and 
\begin{equation}
V_{wall} = -\frac{k_z}{2} \delta
\end{equation}
Thus, 
\begin{equation}
V_w = -\frac{\delta}{G}
\end{equation}
for Dominics value of $\delta = 0.0036$, I have $V_w = -80$ and the highest
I ever used was $V_w = 105$ as John said in our conversation. So this is in agreement (I thought there a factor or 2
issue but I was wrong) Also, note the sign causes his 
crystal to have an equilbrium structure to be elongated along the x-axis (in the rotating frame) while mine
is elongated along the y-axis. This caused a brief confusion for me because inputting my equilibrium structure
in to the simulation gave particle rearrangment almost immediately. After the fix, my equilibrium positions
have little motion in the rotating frame.




\end{document}
